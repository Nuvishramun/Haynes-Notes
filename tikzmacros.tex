
\def\columnsep{\pgfmatrixcolumnsep} %\pgfmatrixcolumnsep is a mouthful.

\tikzset{stable/.style={postaction={decorate},decoration={markings,mark=at position 0.5 with {\arrow[yscale=1]{|}}}}} % Stable arrows
% yscale changes length of crossbar. I'm not sure I'm currently happy with these
% it's annoying that it gets in the way of the label, which also likes to go in the middle. Maybe move the stable slash to the side a bit or something?

\tikzset{
    wavy/.style={/tikz/decorate,/tikz/decoration={
        snake,
        segment length=16\pgflinewidth,
        amplitude=2.3\pgflinewidth,
        post=lineto, post length=4\pgflinewidth,
        pre=lineto, pre length=4\pgflinewidth
    }},
    %
    % This is a function from tikzcd. Just copying it into the /tikz/ namespace. it isn't used anywhere right now though.
    description/.style={
        /tikz/anchor=center,
        /tikz/fill=\pgfkeysvalueof{/tikz/commutative diagrams/background color}
    }
}

\pgfkeys{cm left hook/.tip={Hooks[width=+0pt 16.8,length=+0pt 4.6,harpoon,line cap=round]}} % make hooks more prominent.



% Makes the \xslashedrightarrow for a labeled \stableto

\pgfdeclareshape{slash underlined}
{
  \inheritsavedanchors[from=rectangle] % this is nearly a circle
  \inheritanchorborder[from=rectangle]
  \inheritanchor[from=rectangle]{north}
  \inheritanchor[from=rectangle]{north west}
  \inheritanchor[from=rectangle]{north east}
  \inheritanchor[from=rectangle]{center}
  \inheritanchor[from=rectangle]{west}
  \inheritanchor[from=rectangle]{east}
  \inheritanchor[from=rectangle]{mid}
  \inheritanchor[from=rectangle]{mid west}
  \inheritanchor[from=rectangle]{mid east}
  \inheritanchor[from=rectangle]{base}
  \inheritanchor[from=rectangle]{base west}
  \inheritanchor[from=rectangle]{base east}
  \inheritanchor[from=rectangle]{south}
  \inheritanchor[from=rectangle]{south west}
  \inheritanchor[from=rectangle]{south east}
  \inheritanchorborder[from=rectangle]
  \foregroundpath{
% store lower right in xa/ya and upper right in xb/yb
    \southwest \pgf@xa=\pgf@x \pgf@ya=\pgf@y
    \northeast \pgf@xb=\pgf@x \pgf@yb=\pgf@y
    \pgf@xc=\pgf@xa
    \advance\pgf@xc by .5\pgf@xb
    \pgf@yc=\pgf@ya
    \advance\pgf@xc by -1.3pt
    \advance\pgf@yc by -1.8pt
    \pgfpathmoveto{\pgfqpoint{\pgf@xc}{\pgf@yc}}
    \advance\pgf@xc by  2.6pt
    \advance\pgf@yc by  3.6pt
    \pgfpathlineto{\pgfqpoint{\pgf@xc}{\pgf@yc}}
    \pgfpathmoveto{\pgfqpoint{\pgf@xa}{\pgf@ya}}
    \pgfpathlineto{\pgfqpoint{\pgf@xb}{\pgf@ya}}
 }
}
% I don't think this is used anywhere
\tikzset{nomorepostaction/.code=\let\tikz@postactions\pgfutil@empty}

\newcommand\xslashedrightarrow[2][]{%
  \mathrel{\tikz[baseline=-.7ex] \path node[slash underlined,draw,->,anchor=south] {\(\scriptstyle #2\)} node[anchor=north] {\(\scriptstyle #1\)};}}


%%
\newcommand{\gettikzxy}[3]{%
  \tikz@scan@one@point\pgfutil@firstofone#1\relax
  \edef#2{\the\pgf@x}%
  \edef#3{\the\pgf@y}%
}

% These four macros taken from Tomas M. Trzeciak's http://www.texample.net/tikz/examples/map-projections/
% (with some improvements)
% Used for drawing the spheres with tangent vectors
\makeatletter

\newcommand\pgfmathsinandcos[2]{%
  \@xp\pgfmathsetmacro\csname sin#1\endcsname{sin(#2)}%
  \@xp\pgfmathsetmacro\csname cos#1\endcsname{cos(#2)}%
}

\newcommand\LongitudePlane[3][current plane]{%
  \pgfmathsinandcos{El}{#2} % elevation
  \pgfmathsinandcos{t}{#3} % azimuth
  \tikzset{#1/.style={cm={\cost,\sint*\sinEl,0,\cosEl,(0,0)}}}
}
\newcommand\LatitudePlane[3][current plane]{%
  \pgfmathsinandcos{El}{#2} % elevation
  \pgfmathsinandcos{t}{#3} % latitude
  \tikzset{#1/.style={cm={\cost,0,0,\cost*\sinEl,(0,\cosEl*\sint)}}} %
}
\newcommand\DrawLongitudeCircle[2][1]{
  \LongitudePlane{\angEl}{#2}
  \tikzset{current plane/.prefix style={scale=#1}}
   % angle of "visibility"
  \pgfmathsetmacro\angVis{atan(sin(#2)*cos(\angEl)/sin(\angEl))} %
  \draw[current plane, name path=long] (\angVis:1) arc (\angVis:\angVis+180:1);
  \draw[current plane,dashed] (\angVis-180:1) arc (\angVis-180:\angVis:1);
}

\newcommand\DrawLatitudeCircle[2][1]{
  \LatitudePlane{\angEl}{#2}
  \tikzset{current plane/.prefix style={scale=#1}}
  \pgfmathsetmacro\sinVis{sin(#2)/cos(#2)*sin(\angEl)/cos(\angEl)}
  % angle of "visibility"
  \pgfmathsetmacro\angVis{asin(min(1,max(\sinVis,-1)))}
  \draw[current plane,black, name path=lat] (\angVis:1) arc (\angVis:-\angVis-180:1);
  \draw[current plane,dashed] (180-\angVis:1) arc (180-\angVis:\angVis:1);
}


% Hopf diagram

\newcommand{\HopfDiagram}[6]{
    %#1: left  bar color -> green
    %#2: right bar color -> green
    %#3: accepts: BlackBeard;
    %             WhiteBeard;   Any of these three will produce SZ,
    %             ArrowBeard;     with various colourings of the arrows
    %#4: x-offset
    %#5: y-offset
    %#6: accepts: cone;         Includes the cone on X*Y
    %             nojoin;       Does not display any X or Y
    %             BottomBar
    %             LeftBar
    \ifthenelse{\equal{#1}{on}}
    {\colorlet{LeftBar}{green}}
    {\colorlet{LeftBar}{black}}
    \ifthenelse{\equal{#2}{on}}
    {\colorlet{BottomBar}{green}}
    {\colorlet{BottomBar}{black}}
    \ifthenelse{\equal{#1}{on}\OR \equal{#2}{on}}
    {\colorlet{TheDot}{green}}
    {\colorlet{TheDot}{black}}
    \ifthenelse{\equal{#3}{BlackBeard}}
    {\colorlet{GlueColor}{black}}
    {\colorlet{GlueColor}{red}}
    \ifthenelse{\equal{#3}{WhiteBeard}}
    {\colorlet{GlueColor}{white}}
    {}

    \ifthenelse{\equal{#6}{cone}}
    {\foreach \i in {0,...,9}
    {\draw (#4+0,.1*\i+#5) -- (#4+1-.1*\i,1+#5);}
    \foreach \i in {1,...,9}
    {\draw (#4+.1*\i,0+#5) -- (#4+1,1-.1*\i+#5);}
    \draw (#4+1,0+#5) -- (#4+1,1+#5);
    \draw (#4+0,1+#5) -- (#4+1,1+#5);
    }{}

    \ifthenelse{\equal{#3}{WhiteBeard}}{}
    {
    \ifthenelse{\equal{#6}{nojoin}}{}
    {
    \ifthenelse{\equal{#6}{BottomBar}}{}
    {\draw[ultra thick,LeftBar] (#4+0,0+#5) -- (#4+0,1+#5);}
    \ifthenelse{\equal{#6}{LeftBar}}{}
    {\draw[ultra thick,BottomBar] (#4+0,0+#5) -- (#4+1,0+#5)};
    \fill[TheDot] (#4+0,0+#5) circle (3.42pt);
    }
    }

    \ifthenelse{\equal{#3}{ArrowBeard}\OR \equal{#3}{BlackBeard}\OR \equal{#3}{WhiteBeard}}
    {
    \ifthenelse{\equal{#6}{BottomBar}}{}
    {\draw[ultra thick,LeftBar] (#4+-.5,-.5+#5) -- (#4+-.5,1+#5);}
    \ifthenelse{\equal{#6}{LeftBar}}{}
    {\draw[ultra thick,BottomBar] (#4+-.5,-.5+#5) -- (#4+1,-.5+#5);}
    \fill[TheDot] (#4+-.5,-.5+#5) circle (3.4pt);
    \foreach \i in {0,2,4,6,8,10}
    {\draw[->,GlueColor] (#4+-.1+.11*\i,-.1+#5) -- (#4+-.4+.14*\i,-.4+#5);}
    \foreach \i in {2,4,6,8,10}
    {\draw[->,GlueColor] (#4+-.1,-.1+.11*\i+#5) -- (#4+-.4,-.4+.14*\i+#5);}
    }{}%end \foreach then end \ifthenelse
} 